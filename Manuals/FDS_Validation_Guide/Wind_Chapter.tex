% !TEX root = FDS_Validation_Guide.tex

\chapter{Wind Engineering and Atmospheric Dispersion}

This chapter presents results of simulations of wind over structures and atmospheric dispersion, all involving a simplified atmospheric boundary layer model in FDS.


\section{UWO Wind Tunnel Experiments}

A description of the UWO Wind Tunnel Experiments is included in Section~\ref{UWO_Wind_Tunnel_Description}.

Figures~\ref{UWO_Test_7_pressure_coefficients_180_1} through \ref{UWO_Test_7_pressure_coefficients_270_2} show comparisons of measured and predicted mean, rms, minimum and maximum values of the pressure coefficients on the surface of a 1:100 scale model of a building in a wind tunnel. The comparison is made for two wind directions. For 180$^\circ$, the windward side of the scale model is its shorter side, and pressure coefficients are compared along lines that are approximately 0~cm, 3~cm, and 4~cm from the axial centerline. The width of the scale model is 9~cm. For 270$^\circ$, the windward side of the scale model is its longer side, and pressure coefficients are compared along lines that are approximately 1~cm, 4~cm, and 7~cm from the centerline. The length of the scale model is 14~cm. The ``side'' profiles are along the side of the scale model, approximately halfway up the height. The discontinuities in the plots represent the transition from the windward side, to the roof, to the leeward side. The side plots do not include windward or leeward side data.

The simulations are run for 5~s whereas the experiments were run for 100~s. The minimum and maximum values for the 5~s simulation are extrapolated so that they may be compared to the measured min and max for the 100~s experiment. The procedure is described in the FDS User's Guide~\cite{FDS_Users_Guide}, in the section describing the {\ct TEMPORAL\_STATISTIC} {\ct 'MIN'} and {\ct 'MAX'}.

\begin{figure}[p]
\begin{tabular*}{\textwidth}{l@{\extracolsep{\fill}}r}
\includegraphics[height=2.1in]{SCRIPT_FIGURES/UWO_Wind_Tunnel/UWO_Cp_mean_180_p0009} &
\includegraphics[height=2.1in]{SCRIPT_FIGURES/UWO_Wind_Tunnel/UWO_Cp_rms_180_p0009} \\
\includegraphics[height=2.1in]{SCRIPT_FIGURES/UWO_Wind_Tunnel/UWO_Cp_min_180_p0009} &
\includegraphics[height=2.1in]{SCRIPT_FIGURES/UWO_Wind_Tunnel/UWO_Cp_max_180_p0009} \\
\includegraphics[height=2.1in]{SCRIPT_FIGURES/UWO_Wind_Tunnel/UWO_Cp_mean_180_p0296} &
\includegraphics[height=2.1in]{SCRIPT_FIGURES/UWO_Wind_Tunnel/UWO_Cp_rms_180_p0296} \\
\includegraphics[height=2.1in]{SCRIPT_FIGURES/UWO_Wind_Tunnel/UWO_Cp_min_180_p0296} &
\includegraphics[height=2.1in]{SCRIPT_FIGURES/UWO_Wind_Tunnel/UWO_Cp_max_180_p0296}
\end{tabular*}
\caption[UWO Wind Tunnel Test 7 Pressure Coefficients, 180\si{\degree}]{UWO Wind Tunnel Test 7 mean pressure coefficients, 180\si{\degree} wind direction.}
\label{UWO_Test_7_pressure_coefficients_180_1}
\end{figure}

\begin{figure}[p]
\begin{tabular*}{\textwidth}{l@{\extracolsep{\fill}}r}
\includegraphics[height=2.1in]{SCRIPT_FIGURES/UWO_Wind_Tunnel/UWO_Cp_mean_180_p0437} &
\includegraphics[height=2.1in]{SCRIPT_FIGURES/UWO_Wind_Tunnel/UWO_Cp_rms_180_p0437} \\
\includegraphics[height=2.1in]{SCRIPT_FIGURES/UWO_Wind_Tunnel/UWO_Cp_min_180_p0437} &
\includegraphics[height=2.1in]{SCRIPT_FIGURES/UWO_Wind_Tunnel/UWO_Cp_max_180_p0437} \\
\includegraphics[height=2.1in]{SCRIPT_FIGURES/UWO_Wind_Tunnel/UWO_Cp_mean_180_side} &
\includegraphics[height=2.1in]{SCRIPT_FIGURES/UWO_Wind_Tunnel/UWO_Cp_rms_180_side}  \\
\includegraphics[height=2.1in]{SCRIPT_FIGURES/UWO_Wind_Tunnel/UWO_Cp_min_180_side} &
\includegraphics[height=2.1in]{SCRIPT_FIGURES/UWO_Wind_Tunnel/UWO_Cp_max_180_side}
\end{tabular*}
\caption[UWO Wind Tunnel Test 7 Pressure Coefficients, 180\si{\degree}]{UWO Wind Tunnel Test 7 mean pressure coefficients, 180\si{\degree} wind direction.}
\label{UWO_Test_7_pressure_coefficients_180_2}
\end{figure}

\begin{figure}[p]
\begin{tabular*}{\textwidth}{l@{\extracolsep{\fill}}r}
\includegraphics[height=2.1in]{SCRIPT_FIGURES/UWO_Wind_Tunnel/UWO_Cp_mean_270_p0067} &
\includegraphics[height=2.1in]{SCRIPT_FIGURES/UWO_Wind_Tunnel/UWO_Cp_rms_270_p0067} \\
\includegraphics[height=2.1in]{SCRIPT_FIGURES/UWO_Wind_Tunnel/UWO_Cp_min_270_p0067} &
\includegraphics[height=2.1in]{SCRIPT_FIGURES/UWO_Wind_Tunnel/UWO_Cp_max_270_p0067} \\
\includegraphics[height=2.1in]{SCRIPT_FIGURES/UWO_Wind_Tunnel/UWO_Cp_mean_270_p0368} &
\includegraphics[height=2.1in]{SCRIPT_FIGURES/UWO_Wind_Tunnel/UWO_Cp_rms_270_p0368} \\
\includegraphics[height=2.1in]{SCRIPT_FIGURES/UWO_Wind_Tunnel/UWO_Cp_min_270_p0368} &
\includegraphics[height=2.1in]{SCRIPT_FIGURES/UWO_Wind_Tunnel/UWO_Cp_max_270_p0368}
\end{tabular*}
\caption[UWO Wind Tunnel Test 7 Pressure Coefficients, 270\si{\degree}]{UWO Wind Tunnel Test 7 mean pressure coefficients, 270\si{\degree} wind direction.}
\label{UWO_Test_7_pressure_coefficients_270_1}
\end{figure}

\begin{figure}[p]
\begin{tabular*}{\textwidth}{l@{\extracolsep{\fill}}r}
\includegraphics[height=2.1in]{SCRIPT_FIGURES/UWO_Wind_Tunnel/UWO_Cp_mean_270_p0669} &
\includegraphics[height=2.1in]{SCRIPT_FIGURES/UWO_Wind_Tunnel/UWO_Cp_rms_270_p0669} \\
\includegraphics[height=2.1in]{SCRIPT_FIGURES/UWO_Wind_Tunnel/UWO_Cp_min_270_p0669} &
\includegraphics[height=2.1in]{SCRIPT_FIGURES/UWO_Wind_Tunnel/UWO_Cp_max_270_p0669} \\
\includegraphics[height=2.1in]{SCRIPT_FIGURES/UWO_Wind_Tunnel/UWO_Cp_mean_270_side} &
\includegraphics[height=2.1in]{SCRIPT_FIGURES/UWO_Wind_Tunnel/UWO_Cp_rms_270_side}  \\
\includegraphics[height=2.1in]{SCRIPT_FIGURES/UWO_Wind_Tunnel/UWO_Cp_min_270_side} &
\includegraphics[height=2.1in]{SCRIPT_FIGURES/UWO_Wind_Tunnel/UWO_Cp_max_270_side}
\end{tabular*}
\caption[UWO Wind Tunnel Test 7 Pressure Coefficients, 270\si{\degree}]{UWO Wind Tunnel Test 7 mean pressure coefficients, 270\si{\degree} wind direction.}
\label{UWO_Test_7_pressure_coefficients_270_2}
\end{figure}




\section{LNG Dispersion Experiments}
\label{Atmospheric Dispersion}

Details of the numerical modeling of these experiments is found in Section~\ref{LNG_Dispersion_Description}.

Figure~\ref{LNG_Dispersion_Burro_profiles} through Fig.~\ref{LNG_Dispersion_MaplinSands_profiles} display the measured velocity and temperature profiles, the corresponding Monin-Obukhov profiles that serve as initial and boundary conditions for FDS, and the resulting time-averaged profiles from the FDS simulations.

Figures~\ref{LNG_Dispersion_1}--\ref{LNG_Dispersion_2} compare measured and predicted downwind concentrations of natural gas originating from spills of liquefied natural gas (LNG) on water. In each case, the measured values are short-time (1~s to 3~s) averages of sensors positioned in arcs at discrete distances downwind of the spill site. For each arc, the maximum value is chosen. The processing of the FDS results follows the same procedure. The sensors were generally located a few meters off the relatively dry, flat terrain.

\newpage

\begin{figure}[p]
\begin{tabular*}{\textwidth}{l@{\extracolsep{\fill}}r}
\includegraphics[height=2.1in]{SCRIPT_FIGURES/LNG_Dispersion/Burro3_vel} &
\includegraphics[height=2.1in]{SCRIPT_FIGURES/LNG_Dispersion/Burro3_tmp} \\
\includegraphics[height=2.1in]{SCRIPT_FIGURES/LNG_Dispersion/Burro7_vel} &
\includegraphics[height=2.1in]{SCRIPT_FIGURES/LNG_Dispersion/Burro7_tmp} \\
\includegraphics[height=2.1in]{SCRIPT_FIGURES/LNG_Dispersion/Burro8_vel} &
\includegraphics[height=2.1in]{SCRIPT_FIGURES/LNG_Dispersion/Burro8_tmp} \\
\includegraphics[height=2.1in]{SCRIPT_FIGURES/LNG_Dispersion/Burro9_vel} &
\includegraphics[height=2.1in]{SCRIPT_FIGURES/LNG_Dispersion/Burro9_tmp}
\end{tabular*}
\caption[LNG Dispersion experiments, Burro velocity and temperature profiles]{LNG Dispersion experiments, Burro velocity and temperature profiles.}
\label{LNG_Dispersion_Burro_profiles}
\end{figure}

\begin{figure}[p]
\begin{tabular*}{\textwidth}{l@{\extracolsep{\fill}}r}
\includegraphics[height=2.1in]{SCRIPT_FIGURES/LNG_Dispersion/Coyote3_vel} &
\includegraphics[height=2.1in]{SCRIPT_FIGURES/LNG_Dispersion/Coyote3_tmp} \\
\includegraphics[height=2.1in]{SCRIPT_FIGURES/LNG_Dispersion/Coyote5_vel} &
\includegraphics[height=2.1in]{SCRIPT_FIGURES/LNG_Dispersion/Coyote5_tmp} \\
\includegraphics[height=2.1in]{SCRIPT_FIGURES/LNG_Dispersion/Coyote6_vel} &
\includegraphics[height=2.1in]{SCRIPT_FIGURES/LNG_Dispersion/Coyote6_tmp}
\end{tabular*}
\caption[LNG Dispersion experiments, Coyote velocity and temperature profiles]{LNG Dispersion experiments, Coyote velocity and temperature profiles.}
\label{LNG_Dispersion_Coyote_profiles}
\end{figure}

\begin{figure}[p]
\begin{tabular*}{\textwidth}{l@{\extracolsep{\fill}}r}
\includegraphics[height=2.1in]{SCRIPT_FIGURES/LNG_Dispersion/Falcon1_vel} &
\includegraphics[height=2.1in]{SCRIPT_FIGURES/LNG_Dispersion/Falcon1_tmp} \\
\includegraphics[height=2.1in]{SCRIPT_FIGURES/LNG_Dispersion/Falcon3_vel} &
\includegraphics[height=2.1in]{SCRIPT_FIGURES/LNG_Dispersion/Falcon3_tmp} \\
\includegraphics[height=2.1in]{SCRIPT_FIGURES/LNG_Dispersion/Falcon4_vel} &
\includegraphics[height=2.1in]{SCRIPT_FIGURES/LNG_Dispersion/Falcon4_tmp}
\end{tabular*}
\caption[LNG Dispersion experiments, Falcon velocity and temperature profiles]{LNG Dispersion experiments, Falcon velocity and temperature profiles.}
\label{LNG_Dispersion_Falcon_profiles}
\end{figure}

\begin{figure}[p]
\begin{tabular*}{\textwidth}{l@{\extracolsep{\fill}}r}
\includegraphics[height=2.1in]{SCRIPT_FIGURES/LNG_Dispersion/MaplinSands27_vel} &
\includegraphics[height=2.1in]{SCRIPT_FIGURES/LNG_Dispersion/MaplinSands27_tmp} \\
\includegraphics[height=2.1in]{SCRIPT_FIGURES/LNG_Dispersion/MaplinSands34_vel} &
\includegraphics[height=2.1in]{SCRIPT_FIGURES/LNG_Dispersion/MaplinSands34_tmp} \\
\includegraphics[height=2.1in]{SCRIPT_FIGURES/LNG_Dispersion/MaplinSands35_vel} &
\includegraphics[height=2.1in]{SCRIPT_FIGURES/LNG_Dispersion/MaplinSands35_tmp}
\end{tabular*}
\caption[LNG Dispersion experiments, Maplin Sands velocity and temperature profiles]{LNG Dispersion experiments, Maplin Sands velocity and temperature profiles.}
\label{LNG_Dispersion_MaplinSands_profiles}
\end{figure}


\begin{figure}[p]
\begin{tabular*}{\textwidth}{l@{\extracolsep{\fill}}r}
\includegraphics[height=2.1in]{SCRIPT_FIGURES/LNG_Dispersion/Burro3} &
\includegraphics[height=2.1in]{SCRIPT_FIGURES/LNG_Dispersion/Burro7} \\
\includegraphics[height=2.1in]{SCRIPT_FIGURES/LNG_Dispersion/Burro8} &
\includegraphics[height=2.1in]{SCRIPT_FIGURES/LNG_Dispersion/Burro9} \\
\includegraphics[height=2.1in]{SCRIPT_FIGURES/LNG_Dispersion/Coyote3} &
\includegraphics[height=2.1in]{SCRIPT_FIGURES/LNG_Dispersion/Coyote5} \\
\multicolumn{2}{c}{\includegraphics[height=2.1in]{SCRIPT_FIGURES/LNG_Dispersion/Coyote6}}
\end{tabular*}
\caption[LNG Dispersion experiments, Burro and Coyote]{LNG Dispersion experiments, Burro and Coyote.}
\label{LNG_Dispersion_1}
\end{figure}

\begin{figure}[p]
\begin{tabular*}{\textwidth}{l@{\extracolsep{\fill}}r}
\includegraphics[height=2.1in]{SCRIPT_FIGURES/LNG_Dispersion/Falcon1} &
\includegraphics[height=2.1in]{SCRIPT_FIGURES/LNG_Dispersion/Falcon3} \\
\multicolumn{2}{c}{\includegraphics[height=2.1in]{SCRIPT_FIGURES/LNG_Dispersion/Falcon4}} \\
\includegraphics[height=2.1in]{SCRIPT_FIGURES/LNG_Dispersion/MaplinSands27} &
\includegraphics[height=2.1in]{SCRIPT_FIGURES/LNG_Dispersion/MaplinSands34} \\
\multicolumn{2}{c}{\includegraphics[height=2.1in]{SCRIPT_FIGURES/LNG_Dispersion/MaplinSands35}}
\end{tabular*}
\caption[LNG Dispersion experiments, Falson and Maplin Sands]{LNG Dispersion experiments, Falcon and Maplin Sands.}
\label{LNG_Dispersion_2}
\end{figure}

\begin{figure}[p]
\begin{center}
\begin{tabular}{c}
\includegraphics[width=6.0in]{SCRIPT_FIGURES/ScatterPlots/FDS_Atmospheric_Dispersion}
\end{tabular}
\end{center}
\caption[Summary of LNG Dispersion predictions]{Summary of LNG Dispersion predictions. Note that the dashed black lines denote plus/minus a factor of two from the measured values. The red dashed lines represent two relative standard deviations about the solid red line, the model average.}
\label{Summary_LNG_Dispersion}
\end{figure}













